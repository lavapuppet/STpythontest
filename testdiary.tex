\documentclass[a4paper,11pt]{article}
\usepackage{fullpage}
\usepackage[
    backend=biber,
    style=ieee,
    sorting=nyt,
    autolang=other
]{biblatex}
\title{\textbf{Software Testing of \\ Numpy Linear Algebra Library\\
        by Team~$7$                                   % Replace t by team number
}
}

\author{Regina \and Suraj \and Johan \and Kannan}  % Replace by your name(s)

    \date{\today}

    \renewcommand{\thesubsection}{\thesection.\Alph{subsection}}

    \begin{document}

\section{Documentation for linalg.dot}:
For 2-D arrays it is equivalent to matrix multiplication, and for 1-D arrays to inner product of vectors (without complex conjugation). For N dimensions it is a sum product over the last axis of a and the second-to-last of b:

    dot(a, b)[i,j,k,m] = sum(a[i,j,:] * b[k,:,m])
    takes: two arrays a,b
    
documentation for linalg.multidot:
Compute the dot product of two or more arrays in a single function call, while automatically selecting the fastest evaluation order.

multi_dot chains numpy.dot and uses optimal parenthesization of the matrices [R44] [R45]. Depending on the shapes of the matrices, this can speed up the multiplication a lot.

If the first argument is 1-D it is treated as a row vector. If the last argument is 1-D it is treated as a column vector. The other arguments must be 2-D.


documentation for linalg.vdot:


documentation for linalg.inner:

documentation for linalg.outer:

documentation for linalg.matmul:

documentation for linalg.tensordot:

documentation for linalg.einsum:

documentation for linalg.matrix_power:

documentation for linalg.kron:

documentation for linalg.eig:

documentation for linalg.eigh:

documentation for linalg.eigvals:

documentation for linalg.eigvalsh:

documentation for linalg.norm:

documentation for linalg.cond:

documentation for linalg.det:

documentation for linalg.matrix_rank:

documentation for linalg.slogdet:
    
documentation for linalg.trace:

documentation for linalg.det:



\end{document}

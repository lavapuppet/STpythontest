 \documentclass[a4paper,11pt]{article}
\usepackage{fullpage}
\usepackage{amsmath}
\usepackage[boxed]{algorithm}   % drop [boxed] is no box around algorithm wanted
\usepackage[noend]{algorithmic} % drop [noend] if endif, endwhile, etc wanted
\renewcommand{\algorithmiccomment}[1]{\hfill // #1}
\usepackage{graphicx}
\usepackage{listings}
\usepackage{tikz}

\lstdefinelanguage{Python}{
keywords={typeof, null, catch, switch, in, int, str, float, self},
keywordstyle=\color{ForestGreen}\bfseries,
ndkeywords={return,class,if,elif,endif,while, do, else, True, False, def},
ndkeywordstyle=\color{BrickRed}\bfseries,
identifierstyle=\color{black},
sensitive =false,
comment=[1]{\#}
commentstyle=\color{purple}\ttfamily,
stringstyle=\color{red}\ttfamily,
}


\usepackage[
    backend=biber,
    style=ieee,
    sorting=nyt,
    autolang=other
]{biblatex}
\title{\textbf{Software Testing of \\ Numpy Linear Algebra Library\\
        by Team~$7$                                   % Replace t by team number
}
}

\author{Regina \and Suraj \and Johan}  % Replace by your name(s)

    \date{\today}

    \renewcommand{\thesubsection}{\thesection.\Alph{subsection}}


\begin{document}
	
	%\tableofcontents 
	
\section{Black-box Testing}
\subsection{linalg.dot}
\subsubsection{documentation}
For 2-D arrays it is equivalent to matrix multiplication, and for 1-D arrays to inner product of vectors (without complex conjugation). For N dimensions it is a sum product over the last axis of a and the second-to-last of b:

    dot(a, b)[i,j,k,m] = sum(a[i,j,:] * b[k,:,m])
   \paragraph{Paramaters}: two arrays a: array\_like First argument.\\
b : array\_like Second argument.\\
out : ndarray, optional output argument. This must have the exact kind that would be returned if it was not used. In particular, it must have the right type, must be C-contiguous, and its dtype must be the dtype that would be returned for dot(a,b). This is a performance feature. Therefore, if these conditions are not met, an exception is raised, instead of attempting to be flexible.
    \paragraph{Returns}:    output : ndarray\\
Returns the dot product of a and b. If a and b are both scalars or both 1-D arrays then a scalar is returned; otherwise an array is returned. If out is given, then it is returned.
Raises: 
ValueError
If the last dimension of a is not the same size as the second-to-last dimension of b.


\subsubsection{tests}


\subsection{linalg.multidot}
Compute the dot product of two or more arrays in a single function call, while automatically selecting the fastest evaluation order.

multi\_dot chains numpy.dot and uses optimal parenthesization of the matrices [R44] [R45]. Depending on the shapes of the matrices, this can speed up the multiplication a lot.

If the first argument is 1-D it is treated as a row vector. If the last argument is 1-D it is treated as a column vector. The other arguments must be 2-D.
\subsubsection{tests}

\subsection{linalg.vdot}

\subsubsection{documentation}
Return the dot product of two vectors.

The vdot(a, b) function handles complex numbers differently than dot(a, b). If the first argument is complex the complex conjugate of the first argument is used for the calculation of the dot product.

Note that vdot handles multidimensional arrays differently than dot: it does not perform a matrix product, but flattens input arguments to 1-D vectors first. Consequently, it should only be used for vectors.
\subsubsection{tests}

\subsection{linalg.inner}
\subsubsection{documentation}
Inner product of two arrays.

Ordinary inner product of vectors for 1-D arrays (without complex conjugation), in higher dimensions a sum product over the last axes.

\subsubsection{tests}
\subsection{linalg.outer}
\subsubsection{documentation}
Compute the outer product of two vectors.

\paragraph{Paramaters}: 
a : (M,) array\_like
First input vector. Input is flattened if not already 1-dimensional.
b : (N,) array\_like
Second input vector. Input is flattened if not already 1-dimensional.\\
out : (M, N) ndarray, optional
A location where the result is stored\\
\paragraph{Returns}:    
out : (M, N) ndarray
out[i, j] = a[i] * b[j]

\subsubsection{tests}

\subsection{linalg.matmul}
\subsubsection{documentation}
Matrix product of two arrays.\\

The behavior depends on the arguments in the following way.\\

If both arguments are 2-D they are multiplied like conventional matrices.\\
If either argument is N-D, N > 2, it is treated as a stack of matrices residing in the last two indexes and broadcast accordingly.\\
If the first argument is 1-D, it is promoted to a matrix by prepending a 1 to its dimensions. After matrix multiplication the prepended 1 is removed.\\
If the second argument is 1-D, it is promoted to a matrix by appending a 1 to its dimensions. After matrix multiplication the appended 1 is removed.\\
Multiplication by a scalar is not allowed, use * instead. Note that multiplying a stack of matrices with a vector will result in a stack of vectors, but matmul will not recognize it as such.\\

\subsubsection{tests}
\subsection{linalg.tensordot}
\subsubsection{documentation}
Compute tensor dot product along specified axes for arrays >= 1-D.

Given two tensors (arrays of dimension greater than or equal to one), a and b, and an array\_like object containing two array\_like objects, (a\_axes, b\_axes), sum the products of a‘s and b‘s elements (components) over the axes specified by a\_axes and b\_axes. The third argument can be a single non-negative integer\_like scalar, N; if it is such, then the last N dimensions of a and the first N dimensions of b are summed over.

\subsubsection{tests}
%\subsection{linalg.einsum}
%\subsubsection{documentation}


%\subsubsection{tests}
\subsection{linalg.matrix\_power}
\subsubsection{documentation}
Raise a square matrix to the (integer) power n.\\

For positive integers n, the power is computed by repeated matrix squarings and matrix multiplications. If n == 0, the identity matrix of the same shape as M is returned. If n < 0, the inverse is computed and then raised to the abs(n).
\subsubsection{tests}
%\subsection{linalg.kron}
%\subsubsection{documentation}


%\subsubsection{tests}
\subsection{linalg.eig}
Compute the eigenvalues and right eigenvectors of a square array.
\paragraph{Paramaters}: 
a : (..., M, M) array\\
\paragraph{Returns}:    
w : (..., M) array\\
Raises: 
LinAlgError
If the eigenvalue computation does not converge.

\subsubsection{documentation}
\subsubsection{tests}
\subsection{linalg.eigh}

\subsubsection{documentation}
Return the eigenvalues and eigenvectors of a Hermitian or symmetric matrix.

Returns two objects, a 1-D array containing the eigenvalues of a, and a 2-D square array or matrix (depending on the input type) of the corresponding eigenvectors (in columns).
\paragraph{Paramaters}: 
a : (..., M, M) array\\
\paragraph{Returns}:    
w : (..., M) ndarray
The eigenvalues in ascending order, each repeated according to its multiplicity.\\
Raises: 
LinAlgError
If the eigenvalue computation does not converge.\\



\subsubsection{tests}
\subsection{linalg.eigvalsh}
\subsubsection{documentation}
Compute the eigenvalues of a Hermitian or real symmetric matrix.

Main difference from eigh: the eigenvectors are not computed.
\paragraph{Paramaters}: 
a : (..., M, M) array\_like\\
\paragraph{Returns}:    
w : (..., M,) ndarray
The eigenvaues in ascending order, each repeated according to its multiplicity.\\
Raises: 
LinAlgError if the eigenvalue computation does not converge.

\subsubsection{tests}
\subsection{linalg.eigvals}
\subsubsection{documentation}
Compute the eigenvalues of a general matrix.

Main difference between eigvals and eig: the eigenvectors aren’t returned.


\paragraph{Paramaters}: 
a : (..., M, M) array\_like
A complex- or real-valued matrix whose eigenvalues will be computed.\\
\paragraph{Returns}:    
w : (..., M,) ndarray
The eigenvalues, each repeated according to its multiplicity. They are not necessarily ordered, nor are they necessarily real for real matrices.\\
Raises: 
LinAlgError
If the eigenvalue computation does not converge.\\
\subsubsection{tests}
\subsection{linalg.norm}
\subsubsection{documentation}
Matrix or vector norm.

This function is able to return one of eight different matrix norms, or one of an infinite number of vector norms (described below), depending on the value of the ord parameter.\\

\paragraph{Paramaters}: 
x : array\_like
Input array. If axis is None, x must be 1-D or 2-D.
ord : {non-zero int, inf, -inf, ‘fro’, ‘nuc’}, optional
Order of the norm (see table under Notes). inf means numpy’s inf object.
axis : {int, 2-tuple of ints, None}, optional
If axis is an integer, it specifies the axis of x along which to compute the vector norms. If axis is a 2-tuple, it specifies the axes that hold 2-D matrices, and the matrix norms of these matrices are computed. If axis is None then either a vector norm (when x is 1-D) or a matrix norm (when x is 2-D) is returned.
keepdims : bool, optional
If this is set to True, the axes which are normed over are left in the result as dimensions with size one. With this option the result will broadcast correctly against the original x.
New in version 1.10.0.\\

\paragraph{Returns}:    
n : float or ndarray
Norm of the matrix or vector(s).\\

\subsubsection{tests}


\subsection{linalg.matrix\_rank}
\subsubsection{documentation}
Return matrix rank of array using SVD method
Rank of the array is the number of singular values of the array that are
greater than `tol`.
\paragraph{Paramaters}:  M : {(M,), (..., M, N)} array\_like input vector or stack of matrices
tol : (...) array\_like, float, optional
threshold below which SVD values are considered zero. If `tol` is None, and ``S`` is an array with singular values for `M`, and ``eps`` is the epsilon value for datatype of ``S``, then `tol` is set to ``S.max() * max(M.shape) * eps`` Broadcasted against the stack of matrices hermitian : bool, optional If True, `M` is assumed to be Hermitian (symmetric if real-valued),
enabling a more efficient method for finding singular values. Defaults to False.
\paragraph{Returns}: 
\subsubsection{tests}
test\_simple\_case: a simple 3X3 eye matrix is tested \\
\\
test\_scalar: the rank of a scalar input equates to 1. \\
\\
test\_array: same as above but a scalar inside an array. \\
\\
test\_zero\_rank: the rank of a matrix with zeros will equate to 0. \\
\\
test\_1\_dimensional\_matrix: the rank of a 1 dimensinal matrix should be 1. \\
\\
These tests were written to test the functionality of matrix rank with various simple integer inputs. 


\subsection{linalg.det, linalg.slogdet}
\subsubsection{documentation}
Determinants are used to define the characteristic polynomial of a matrix and whether it has a unique solution or not. This function computes the sign and (natural) logarithm of the determinant of an array. A number representing the sign of the determinant. For a real matrix,
this is 1, 0, or -1. For a complex matrix, this is a complex number with absolute value 1 (i.e., it is on the unit circle), or else 0. The determinant is computed via LU factorization using the LAPACK
routine z/dgetrf. The determinant of a 2-D array ``[[a, b], [c, d]]`` is ``ad - bc``. (sign, logdet) = np.linalg.slogdet(a)
\paragraph{Paramaters}: An array or matrix with single, double, complex single or complex double type. 
\paragraph{Returns}: A scalar. 
\subsubsection{tests}
test\_det: This tests that the determinant calculation works according to the above. \\
\\
test\_size\_zero: This tests that the sign of the determinant an empty matrix is a complex number and that the determinant itself is 1. \\
\\
test\_types: This tests that the output type of the determinant is the same as the input type, i.e. single, double, csingle and cdouble. \\
\\
These tests were written to test matrix determinants with various simple inputs. The determinant function was also evaluated with double, complex single and complex double datatypes. 

\subsection{linalg.multidot (Black box tests)}
\subsubsection{documentation}
Compute the dot product of two or more arrays in a single function call, while automatically selecting the fastest evaluation order. `multi\_dot` chains `numpy.dot` and uses optimal parenthesization of the matrices. Depending on the shapes of the matrices, this can speed up the multiplication a lot. If the first argument is 1-D it is treated as a row vector. If the last argument is 1-D it is treated as a column vector. The other arguments must be 2-D.\\
\\
TestCases: Test cases are created so that vectors when multiplied share the same dimensions. When matrices are multiplied they need to be organized so that the first dimension of the first matrix is the same as the second dimension of the second matrix etc. 
\paragraph{Paramaters}: Vectors or matrices. They must be organized so that the first dimension of the first matrix is the same as the second dimension of the second matrix etc. 
\paragraph{Returns}: A vector or matrix whose dimension depends on the inputs. 
\subsubsection{Tests}
test\_three\_inputs\_vectors: This tests the multidot function with three vectors. The assert is the following: assert\_almost\_equal(multi\_dot([A, B]), A.dot(B))\\
\\
test\_three\_inputs\_matrices: This tests the multidot function with three matrices\\
\\
test\_four\_inputs\_matrices: This tests the multidot function with four matrices\\
\\
test\_shape\_vector\_first: This tests the multidot function with a vector with n rows as the first argument followed by three matrices with dimensions n, m and m, n. The shape result sought is the same as the vector, i.e. 1 dimensional with n rows. \\
\\
test\_shape\_vector\_last: This tests the multidot function with a n rows vector as the last argument preceded by three matrices with dimensions m, n and n, m. The shape result sought is m. \\
\\
test\_shape\_vector\_first\_and\_last: This tests the multidot function with n rows vector as the first and last arguments with two matrices with dimensions n, m and m, n in the middle. The shape result sought is () since the result is a scalar. assert\_equal(multi\_dot([A1d, B, C, D1d]).shape, ())\\
\\
test\_types: This runs the test\_three\_inputs\_matrices above using integers, doubles, complex numbers. \\
\\
These tests were written to test the functionality of multidot with various inputs. All test cases are initialized with random values. 

\subsection{Datatype tests}
A separate testclass was created to test linalg functions with various datatypes. The functions were run with values of these datatypes and the output was checked to still be the same datatype. The datatypes used were single, double, complex single and complex double. The functions tested with the datatypes were matrix invariant, eig and eigenvalues for normal and hermitian cases, single value decomposition and determinant.  



\section{White-Box Test}
In this section we aim to use what we can see from the functions themselves to satisfy some coverage
criteria. To evaluate coverage we will use the coverage.py package. This can evaluate both statement and branch coverage and enumerate which statements or branches were not executed.

As our function contains a loop we want to include loop coverage. 

The function we have chosen to white box test is the multi\_dot() function and it's subsidiary functions \_multi\_dot() and multi\_dot\_three().

This function performs the dot product of an array of arrays. It consists of several if/ else statements a recursive loop and several different return options. In our testing we want to ensure that all statments and branches are covered along with coverage of the loops. 



\paragraph{Node Coverage}


For node coverage we have the critereon that our tests cause all statements in the program to be executed. Thus we want to ensure that in our set of tests that all nodes are visited on at least one test path. Figure 1 shows the control flow graph for the functions under test. 



\paragraph{Edge Coverage}

For edge coverage we have the critereon that our tests cause all branches to execute. In this case we have the set of edges \\
\{(1,2),(1,3),(1,4),(4,5),(4,6),(5,6),(6,7),(6,8),(8,9),(8,10),(8,12),(12,10),(10,16),(9,11),(11,13)\\
,(13,14),(13,15),(15,16),(16,17),(16,18),(16,19)\}.
Our test requirements are that every edge is contained in at least one of our test paths.
\\

\paragraph{Loop Coverage}

A loop is covered if in at least one test executed the loop 0 times, if in some test the loop was executed exactly once, and if in some test the body was executed more than once. In the case of this code we can't test it only once so we execute it a minimum number of times i.e. with four arrays.
This is because the function \_multi\_dot is only called when there are more than three arguments. It is a self calling function that iteratively divides the arrays based on a precomputed best order. it stops when the two indices passed are the same so the minimum number of calls is more than one.
\begin{itemize}
\item Zero times - Test 1 - The \_multi\_dot function is not called.  
\item Minimum - Test 5.
\item Many times - test 7 - The loop recurses many times
\end{itemize}


To achieve coverage for these cases we create a set of test paths that include all nodes and edges given above along with paths that execute the loop zero times, the minimum amount of times and many times. 

\begin{enumerate}
\item \{1,2\}  
\item \{(1,3)\}
\item \{1,4,5,6,8,9,11,13,14,16,18\}
\item \{1,4,6,7,8,9,11,13,15,16,18\}
\item \{1,4,5,6,7,8,10,12,10,16,17\}
\item \{1,4,6,8,9,11,13,14,16,19\}
\item \{1,4,5,6,7,8,10,12,10,12,10,16,17\}
\end{enumerate}



\paragraph{Test Path 1}\\

To construct a test case for the first path enumerated above we need to pass an array with fewer than 2 arguments. 
The test test\_multi\_dot\_raises was created to execute path 1. It returns a raises value error.


\paragraph{Test Path 2}\\

To execute path 2 the test test\_multi\_two was created. This path was constructed to have exactly two arguments in the passed array.
This path calls the dot fonction and then returns. It returns the dot product of the two.


\paragraph{Test Path 3}\\

Test path 3 tests three different branches:\\
\begin{itemize}
\item By setting the number of array arguments to 3 we take the third arm of the first branch which brings us into the main body of the program. by choosing exactly 3 arguments we also test the branch calling the multi\_dot\_three function or the branch from 8-9.
\item By not setting the dimension of the last argument to 1 we do not execute the if statement from 6-7.
\item By setting the dimesion of the first argument to be 1 we execute the branches 4-5 and 16-17.
\item The ordering of the arguments dictates which branch is taken within the multi\_dot\_three function. The test test\_multi\_ndim\_10 is ordered so that the branch from 13-14 is taken. 
\end{itemize}


%\paragraph{Test 4: Arguments=3,dimension of last argument = 1}
\paragraph{Test path 4}\\
test case - test\_multi\_ndim\_01\\
Test case 3 tests three different branches:\\
\begin{itemize}
\item By setting the dimesion of the last argument to be 1  we execute the if statement branch from  6-7.
\item By not setting the dimension of the first argument to 1 we do not execute the if statement branch from 4-5.
\item The ordering of the arguments dictates which branch is taken within the multi\_dot\_three function. The arguments of test\_multi\_ndim\_01 are ordered so that the branch from 13-15 is taken. 
\end{itemize}



%\paragraph{Test 5: Arguments$>$3,dimension of first and last argument = 1}
\paragraph{Test Path 5}\\

test case - test\_multi\_ndim\_11\\
Test case 3 tests three different branches:\\
\begin{itemize}
\item By setting the dimesion of the last argument and the last argument to be 1  we execute both if statement branches from 4-5 and  6-7.
\item By having more than 3 arguments we take the branch into the \_multi\_dot fucntion from node 8-10. 
\item By setting the number of arguments to 4 we execute the loop a minimum number of times.   
\item As both the dimesion of the first and last argument are 1 we execute the if statement from 16-17.   
\end{itemize}


%\paragraph{Test 6: Arguments=3,dimension of first and last argument $>$ 1}
\paragraph{Test Path 6}

test case - test\_multi\_ndim\_00\\
This test was created to test the case where neither of the if loops from 4-5 and 6-7 are executed. This also gives that the if statement from 16-19 is executed.  


\paragraph{Test Path 7}
test case - test\_many\_ndim\_11\\

This test was created to test the case where the loop 10-12-10 is executed multiple times.

 This also gives that the if statement from 16-19 is executed.  



%in preamble
\usepackage{tikzpicture}
%in main

  \begin{tikzpicture}[%
    ->,
    shorten >=2pt,
    >=stealth,
    node distance=1cm,
    noname/.style={%
      ellipse,
      minimum width=5em,
      minimum height=3em,
      draw
    }
  ]
    \node[noname] (1)                                             {1};
    \node[noname] (2) [below=of 1]                                {2};
    \node[noname] (4) [node distance=1cm and 3mm,below left=of 2] {4};
    \node[noname] (3) [left=of 4]                                 {3};
    \node[noname] (5) [below=of 4]                                {5};
    \node[noname] (6) [node distance=2cm,right=of 5]              {6};

    \path (1) edge                   node {} (2)
          (2) edge                   node {} (3)
          (2) edge                   node {} (4)
          (2) edge                   node {} (6)
          (3) edge                   node {} (5)
          (4) edge                   node {} (5)
          (5) edge [bend right=20pt] node {} (2);
  \end{tikzpicture}







\section{Appendix}

\subsection{whitebox}

\lstinputlisting{whiteBoxTest.py}


\subsection{linalg.dot}
\begin{algorithm}[H]
	\textbf{class} numpy \textbf{as} np
\\	\textbf{import} unittest
\\
\\ \textbf{class} TestLinAlg(unittest.TestCase):
\\
\\$ ~~~~~~~~ $\textbf{def} setUp(self):	
\\ $ ~~~~~~~~ $''\url{http://gettingsharper.de/2011/11/30/vector-fun-dot-product/}" 
\\
\\ $ ~~~~~~~~ $\textit{Basic Identity Test / Square Test}
\\ $ ~~~~~~~~ $self.array\_1 = [[1, 0], [0, 1]]
\\ $ ~~~~~~~~ $self.array\_2 = [[4, 1], [2, 2]]
\\ $ ~~~~~~~~ $
\\ $ ~~~~~~~~ $\textit{Zero Test}
\\ $ ~~~~~~~~ $self.array\_zero = [0, 0]
\\ $ ~~~~~~~~ $
\\ $ ~~~~~~~~ $\textit{Commutative Test}
\\ $ ~~~~~~~~ $self.array\_com\_1 = [-3.22 , 2.25, -0.13]
\\ $ ~~~~~~~~ $self.array\_com\_2 = [0.0 , -6.7, 10.0]   
\\ $ ~~~~~~~~ $
\\ $ ~~~~~~~~ $\textit{Linear Test}
\\ $ ~~~~~~~~ $self.array\_com\_3 = [12.4, -1.7, 3.15]
\\ $ ~~~~~~~~ $self.scalar = 0.22
\\ $ ~~~~~~~~ $
\\ $ ~~~~~~~~ $\textit{Perpendicular Test}
\\ $ ~~~~~~~~ $self.array\_per\_1 = [2.0 , 1.0, 4.0]
\\ $ ~~~~~~~~ $self.array\_per\_2 = [1.0 , -1.0, -0.25]
\end{algorithm}

\begin{algorithm}[H]
	\textbf{def} test\_dot\_corner(self):
	\\ $ ~~~~~~~~ $actual = np.dot([ ], [ ])
	\\ $ ~~~~~~~~ $expected = False
	\\ $ ~~~~~~~~ $self.assertEqual(actual, expected);
\end{algorithm}

\begin{algorithm}[H]
	\textbf{def} test\_dot\_corner2(self):
	\\ $ ~~~~~~~~ $with self.assertRaises(ValueError):
	\\ $ ~~~~~~~~~~~~~~~~ $actual = np.dot([ ], [1, 2])
\end{algorithm}

\begin{algorithm}[H]
	\textbf{def} test\_dot\_identity(self):
	\\ $ ~~~~~~~~ $actual = np.dot(self.array\_1, self.array\_2)
	\\ $ ~~~~~~~~ $expected = [[4, 1], [2, 2]]
	\\ $ ~~~~~~~~ $self.assertTrue((actual == expected).all())
\end{algorithm}

\begin{algorithm}[H]
	\textbf{def} test\_dot\_zero(self):
	\\ $ ~~~~~~~~ $actual = np.dot(self.array\_zero, self.array\_2)
	\\ $ ~~~~~~~~ $expected = 0
	\\ $ ~~~~~~~~ $self.assertTrue((actual == expected).all())
\end{algorithm}

\begin{algorithm}[H]
    \textbf{def} test\_dot\_commutative(self):
\\ $ ~~~~~~~~ $actual = np.dot(self.array\_com\_1, self.array\_com\_2)
\\ $ ~~~~~~~~ $expected = np.dot(self.array\_com\_2, self.array\_com\_1)
\\ $ ~~~~~~~~ $self.assertTrue((actual == expected).all())
\end{algorithm}

\begin{algorithm}[H]
    \textbf{def} test\_dot\_square(self):
\\ $ ~~~~~~~~ $actual = np.dot(self.array\_2, self.array\_2)
\\ $ ~~~~~~~~ $expected = [[18, 6], [12, 6]]
\\ $ ~~~~~~~~ $self.assertTrue((actual == expected).all())
\end{algorithm}

\begin{algorithm}[H]
	
	\textbf{def} test\_dot\_perpendicuar(self):
	\\ $ ~~~~~~~~ $	actual = np.dot(self.array\_per\_1, self.array\_per\_2)
	\\ $ ~~~~~~~~ $	expected = 0
	\\ $ ~~~~~~~~ $	self.assertTrue((actual == expected).all())
\end{algorithm}


\begin{algorithm}[H]
    \textbf{def} test\_dot\_raises(self):
\\ $ ~~~~~~~~ $with self.assertRaises(ValueError):
\\ $ ~ ~~~~~~~~ ~~~~~~~ $actual = np.dot([2, 2, 3], [2, 1])
\end{algorithm}

\subsection{linalg.vdot}
\begin{algorithm}[H]
\textbf{class} TestLinAlg(unittest.TestCase):
\\ $ ~~~~~~~~ $\textbf{def} setUp(self):
\\ $ ~~~~~~~~~~~~~~~~ $self.array\_a = np.array([[1, 4], [5, 6]])
\\ $ ~~~~~~~~~~~~~~~~ $self.array\_b = np.array([[4, 1], [2, 2]])
\\
\\ $ ~~~~~~~~~~~~~~~~ $self.array\_a\_float = np.array([1.0, 4.5])
\\ $ ~~~~~~~~~~~~~~~~ $self.array\_b\_float = np.array([3.0, 2.5])
\\
\\ $ ~~~~~~~~ $\textbf{def} setupForComplex(self):
\\ $ ~~~~~~~~~~~~~~~~ $self.complex\_a = np.array([1+2j,3+4j])   
\\ $ ~~~~~~~~~~~~~~~~ $self.complex\_b = np.array([5+6j,7+8j])
\\ $ ~~~~~~~~~~~~~~~~ $self.complex\_c = np.array([-5-6j,-7-8j])
\end{algorithm}


\begin{algorithm}[H]
	\textbf{def} test\_vdot\_square(self):
	\\ $ ~~~~~~~~ $actual = np.vdot(self.array\_a, self.array\_a)
	\\ $ ~~~~~~~~ $expected = 78
	\\ $ ~~~~~~~~ $self.assertTrue(actual == expected)
\end{algorithm}

\begin{algorithm}[H]
	\textbf{def} test\_vdot\_complexSquare(self):
	\\ $ ~~~~~~~~ $self.setupForComplex()
	\\
	\\ $ ~~~~~~~~ $actual = np.vdot(self.complex\_a, self.complex\_a)
	\\ $ ~~~~~~~~ $expected = 30+0j
	\\
	\\ $ ~~~~~~~~ $self.assertTrue(actual == expected)
\end{algorithm}

\begin{algorithm}[H]
    \textbf{def} test\_vdot\_normal(self):
\\ $ ~~~~~~~~ $self.setupForComplex()
\\ $ ~~~~~~~~ $actual = np.vdot(self.complex\_a, self.complex\_b)
\\ $ ~~~~~~~~ $expected = 70-8j
\\ $ ~~~~~~~~ $self.assertTrue(actual == expected)
\end{algorithm}

\begin{algorithm}[H]
    \textbf{def} test\_vdot\_com(self):
\\ $ ~~~~~~~~ $self.setupForComplex()
\\ $ ~~~~~~~~ $actual = np.vdot(self.array\_a, self.array\_b)
\\ $ ~~~~~~~~ $expected = np.vdot(self.array\_b, self.array\_a)
\\ $ ~~~~~~~~ $self.assertTrue(actual == expected)
\end{algorithm}

\begin{algorithm}[H]
    \textbf{def} test\_vdot\_negative(self):
\\ $ ~~~~~~~~ $self.setupForComplex()
\\ $ ~~~~~~~~ $actual = np.vdot(self.complex\_c, self.complex\_a)
\\ $ ~~~~~~~~ $expected = -70-8j
\\ $ ~~~~~~~~ $self.assertTrue(actual == expected)
\end{algorithm}

\begin{algorithm}[H]
    \textbf{def} test\_vdot\_float(self):
\\ $ ~~~~~~~~ $actual = np.vdot(self.array\_a\_float, self.array\_b\_float)
\\ $ ~~~~~~~~ $expected = 14.25
\\ $ ~~~~~~~~ $self.assertTrue(actual == expected)
\end{algorithm}

\begin{algorithm}[H]
    \textbf{def} test\_vdot\_empty(self):
\\ $ ~~~~~~~~ $actual = np.vdot([ ],[ ])
\\ $ ~~~~~~~~ $self.assertFalse(actual)
\end{algorithm}

\subsection{linalg.inner}

\begin{algorithm}[H]
    \textbf{def} setUp(self):
\\
\\ $ ~~~~~~~~ $self.array\_a = np.array([1, 2, 3])
\\ $ ~~~~~~~~ $self.array\_b = np.array([0, 1, 0])
\\
\\ $ ~~~~~~~~ $self.array\_a\_float = np.array([1.0, 2.0, 4.5])
\\ $ ~~~~~~~~ $self.array\_b\_float = np.array([3.0, 3.5, 2.5])
\\
\\ $ ~~~~~~~~ $\textit{Inner Product Wolfram} \\$ ~~~~~~~~ $\url{http://mathworld.wolfram.com/InnerProduct.html}
\\ $ ~~~~~~~~ $self.vector\_u = np.array([1,2,3])
\\ $ ~~~~~~~~ $self.vector\_v = np.array([1,2,1])
\\ $ ~~~~~~~~ $self.vector\_w = np.array([4,5,6])
\\ $ ~~~~~~~~ $self.scalar = 5
\\ $ ~~~~~~~~ $self.vector\_zero = np.array([0, 0, 0])
\end{algorithm}


\begin{algorithm}[H]
	\textbf{def} test\_inner\_simple(self):
\\ $ ~~~~~~~~ $	actual = np.inner(self.array\_a, self.array\_b)
\\ $ ~~~~~~~~ $	expected = 2
\\ $ ~~~~~~~~ $	self.assertTrue(actual == expected)
\end{algorithm}

\begin{algorithm}[H]
    \textbf{def} test\_inner\_zero(self):
\\ $ ~~~~~~~~ $actual = np.inner(self.array\_a, [0, 0, 0])
\\ $ ~~~~~~~~ $expected = 0
\\ $ ~~~~~~~~ $self.assertTrue(actual == expected)
\end{algorithm}

\begin{algorithm}[H]
\textbf{def} test\_inner\_float(self):
\\ $ ~~~~~~~~ $actual = np.inner(self.array\_a\_float, self.array\_b\_float)
\\ $ ~~~~~~~~ $expected = 21.25
\\ $ ~~~~~~~~ $self.assertTrue(actual == expected)
\end{algorithm}

\begin{algorithm}[H]
\textbf{def} test\_inner\_prop1(self): 
\\ $ ~~~~~~~~ $actual = np.inner(np.add(self.vector\_u,self.vector\_v), self.vector\_w)
\\ $ ~~~~~~~~ $expected = np.add(np.inner(self.vector\_u,self.vector\_w), np.inner(self.vector\_v,self.vector\_w))
\\ $ ~~~~~~~~ $self.assertTrue(actual == expected)
\end{algorithm}

\begin{algorithm}[H]
\textbf{def} test\_inner\_prop2(self):
\\ $ ~~~~~~~~ $actual = np.inner(self.scalar * self.vector\_v, self.vector\_w)
\\ $ ~~~~~~~~ $expected = self.scalar * np.inner(self.vector\_v, self.vector\_w)
\\ $ ~~~~~~~~ $self.assertTrue(actual == expected)
\end{algorithm}

\begin{algorithm}[H]
\textbf{def} test\_inner\_prop3(self):
\\ $ ~~~~~~~~ $actual = np.inner(self.vector\_v, self.vector\_w)
\\ $ ~~~~~~~~ $expected = np.inner(self.vector\_w, self.vector\_v)
\\ $ ~~~~~~~~ $self.assertTrue(actual == expected)
\end{algorithm}

\begin{algorithm}[H]
\textbf{def} test\_inner\_prop4(self):
\\ $ ~~~~~~~~ $actual = np.inner(self.vector\_zero, self.vector\_zero)
\\ $ ~~~~~~~~ $expected = 0
\\ $ ~~~~~~~~ $self.assertTrue(actual == expected)
\end{algorithm}

\begin{algorithm}[H]
\textbf{def} test\_inner\_raises(self):
\\ $ ~~~~~~~~ $with self.assertRaises(ValueError):
\\ $ ~~~~~~~~~~~~~~~~ $actual = np.inner([2, 2, 3], [2, 1])
\end{algorithm}

%\subsection{linalg.det, linalg.slogdet}

\subsection{test\_multi\_dot}	
\begin{figure}[H]
	\centering
	\includegraphics[width=0.6\textwidth]{snippets/multi_dot/1CASES.PNG}
	\includegraphics[width=0.6\textwidth]{snippets/multi_dot/2.PNG}
	\includegraphics[width=0.6\textwidth]{snippets/multi_dot/3.PNG}
	\includegraphics[width=0.5\textwidth]{snippets/multi_dot/4.PNG}

\end{figure}

\subsection{test\_matrix\_rank}	
\begin{figure}[h]
	\centering
	\includegraphics[width=0.70\textwidth]{snippets/rank/1.PNG}
\end{figure}

\subsection{test\_matrix\_determinant}	
\begin{figure}[h]
	\centering
	\includegraphics[width=0.70\textwidth]{snippets/Det/1.PNG}
\end{figure}

\subsection{test\_datatypes}	
\begin{figure}[h]
	\centering
	\includegraphics[width=0.70\textwidth]{snippets/datatypes/2.PNG}
	\includegraphics[width=0.70\textwidth]{snippets/datatypes/3.PNG}
	\includegraphics[width=0.70\textwidth]{snippets/datatypes/4.PNG}
\end{figure}

\newpage	
\end{document}
